\documentclass[xcolor=x11names,compress]{beamer}

%% General document %%%%%%%%%%%%%%%%%%%%%%%%%%%%%%%%%%
\usepackage[german,polish,romanian,english]{babel}
\usepackage[T1]{tipa}
\usepackage{ragged2e}
\usepackage{graphicx}
\usepackage{tikz}
\usetikzlibrary{decorations.fractals}
\usepackage{enumitem}
\usepackage{gb4e}
\usepackage{qtree}
\usepackage{soul}
\usepackage{bbding}

%%%%%%%%%%%%%%%%%%%%%%%%%%%%%%%%%%%%%%%%%%%%%%%%%%%%%%


%% Beamer Layout %%%%%%%%%%%%%%%%%%%%%%%%%%%%%%%%%%
\useoutertheme[subsection=false,shadow]{miniframes}
\useinnertheme{default}
\usefonttheme{serif}
\usepackage{palatino}

\setbeamerfont{title like}{shape=\scshape}
\setbeamerfont{frametitle}{shape=\scshape}

\setbeamercolor*{lower separation line head}{bg=DeepSkyBlue4} 
\setbeamercolor*{normal text}{fg=black,bg=white} 
\setbeamercolor*{alerted text}{fg=red} 
\setbeamercolor*{example text}{fg=black} 
\setbeamercolor*{structure}{fg=black} 
 
\setbeamercolor*{palette tertiary}{fg=black,bg=black!10} 
\setbeamercolor*{palette quaternary}{fg=black,bg=black!10} 

\renewcommand{\(}{\begin{columns}}
\renewcommand{\)}{\end{columns}}
\newcommand{\<}[1]{\begin{column}{#1}}
\renewcommand{\>}{\end{column}}
%%%%%%%%%%%%%%%%%%%%%%%%%%%%%%%%%%%%%%%%%%%%%%%%%%


%\documentclass{beamer}
%\usepackage{beamerthemesplit} 
%\usetheme{Goettingen}
%\usecolortheme{seahorse}
\title{Semantic Role Labeling}
\author{Arne Binder, Robert B\"arhold, Enrique Manjavacas}
\date{
	\begin{tikzpicture}[decoration=Koch curve type 2] 
		\draw[DeepSkyBlue4] decorate{ decorate{ decorate{ (0,0) -- (2,0) }}}; 
	\end{tikzpicture}  
	\\
	\vspace{0cm}
	\today
}


\begin{document}

\frame{\titlepage}

%\tableofcontents?

\section[Gliederung]{}
\frame{
\begin{itemize}
	\item<1-> 1.
	\item<1-> 2.
\end{itemize}
}

\section{theoretischer Hintergrund und Motivation}
\frame{
\frametitle{Argumentstruktur von Verben}
Beispielverb - zweiseitige Argumentstruktur
}
\frame{
\frametitle{thematische Rollen}
Abstraktion \"uber die semantische Seite der Argumentstruktur
}

\frame{
\frametitle{Frames und Frame elements}
Situationsverankerte Abstraktion \"uber die Argumentstruktur
Frame element gleich Rolle
target / praedikat
}
\frame{
\frametitle{Motivation}
%Anwendungsbeispiele 
%Informationextraktion,...
%Warum Rollen? theoretisches Beispiel
%nicht aufschreiben, aber sagen - the linguists scarcity problem, zu viele daten zu wenige linguisten

}

\section{Problemstellung}


\frame{
	\frametitle{Problemstellung}
Automatische Bestimmung von Rollen 
Lassen sich semantische Rollen anhand von lexikalischen und syntaktischen Informationen eines Satzes automatisch bestimmen?
%box rum blauer Rahmen

}
\frame{
\frametitle{Corpus: Salsa}
information zu salsa - Lust zu schreiben?
annotierung genau beschreiben (saetzengrenzen etc...)
}
\section{Umsetzung}
\frame{
\frametitle{Einschr\"ankung}
top 10 rollen
keine Frames
}
\frame{
\frametitle{Herangehensweise}
Ansatz von jurafsky(und gildea)
Classifier - naive bayes classifier
features - anlehnend an jurafsky und gildea (aufzaehlen)
}
\frame{
\frametitle{Probleme}
Deutsch erlaubt wesentlich freiere Verschiebung von Konstituenten im Satz - Subjekt nicht auf die Vorfeldposition beschr\"ankt....
Corpusannotation - 
1. Head nicht \"uberall definiert. Negra Tag Set vorentscheidungen. 
2. ich ging nach haus' und dr\"uckte die maus (sec-edges)
3. target identifizieren. gegebenenfalls mehrere woerter

}
\frame{
\frametitle{SRL-Ablauf}
1. Corpus einlesen - aufgeteilt nach Saetzen
2. interne corpusrepraesentation mit informationen anreichern
%regelbasiert werden heads gesetzt pfade zum root ermitteln..............
3. model tranieren [h\"aufigkeiten auszaehlen und normieren]
%leerzeile
4. nicht frame annotiertes Corpus einlesen
5. konstituenten klassifizieren mithilfe des models.
[6. evaluieren]
}
\frame{
\frametitle{Ablauf des Trainings und der Klassifikation}
a)Training
wird alles gezaehlt und target lemmata extrahiert
b)Klassifikationen
Satz > Target lemma suchen > alle Konstituenten klassifizieren \& Frame bzg Target lemma speichern
}

\section{Evaluieren}
\frame{
da wir eine feste liste von target lemmata nutzen und nicht auf ungesehene targets abstrahieren koennen 
beruecksichtigen wird bei der evaluation nur frame elements mit dem bezug auf ein (im Model) bekanntes target. 
counts erklaeren und ausgeben 
}
\frame{
\frametitle{Kreuzvalidierung}
fuenffach plus ergebnisse

}









































\frame{
\frametitle{What are semantic  roles?}
\begin{itemize}
	\item<2->
			\begin{xlist}
			\ex
			\gll Peter gibt Karl Geld.\\
			\footnotesize
			\textcolor{blue}{\textit{SOURCE}}
			\footnotesize
			{}
			\footnotesize
			\textcolor{blue}{\textit{RECIPIENT} }
			\footnotesize
			\textcolor{blue}{\textit{THEMA}}
			\\
			\vspace{1cm}
	\item<3->
			
			\gll Karl bekommt Geld von Peter.\\
			\footnotesize
			\textcolor{blue}{\textit{RECIPIENT}}
			\footnotesize
			{}
			\footnotesize
			\textcolor{blue}{\textit{THEMA}}
			\footnotesize
			\textcolor{blue}{\textit{SOURCE}}
			\\
			\end{xlist}

			
\end{itemize}
}



\section[Why Semantic Roles?]{}
\frame{
\frametitle{Why Semantic Roles?}
\begin{itemize}
	\item<1-> \begin{block}{\textbf{Information Extraction}}
	\begin{itemize}
		\item "Paul hat seine Mutter get\"otet."
	\end{itemize}
	\end{block}
	\vspace{1cm}
	\item<2-> \begin{block}{\textbf{Metaphernanalyse}}
	\begin{itemize}
		\item "Paul hat seine Mutter unter die Erde gebracht."
	\end{itemize}
	\end{block}
\end{itemize}
}

\section[Our goal]{}
\frame{
\frametitle{Our goal}
\begin{itemize}
	\item<1-> Automatic Semantic Role Labeling
	\item<2-> Language: German
	\item<3-> Corpus: SALSA 2.0 (based on TIGER Corpus \& FrameNet)
	\item<4-> Framework: LingPipe 
	\vspace{1cm}
	\item<5-> \textbf{References:} C.J. Fillmore (Frame Semantics), D. Jurafsky \& D. Gildea (first implementation of a Semantic Role Labeler)
	
\end{itemize}
}



\end{document}
